\chapter{Conclusion}
\label{sec:conclusion}

\section{Global Interpretations of Results}

The analyses presented in Chapters~\ref{sec:bsm_H_to_tau_tau_analysis} and \ref{sec:H_A_to_4_tau_analysis}, although motivated by different physics, are complementary to one another in context of the type X \ac{2HDM}.
The limits on this phase space from the analysis discussed in Chapter~\ref{sec:bsm_H_to_tau_tau_analysis}, are studied using the \textsc{HiggsTools-1} framework~\cite{Bahl:2022igd}.
\textsc{HiggsTools} is a combination of the \textsc{HiggsBounds}, \textsc{HiggsSignals} and \textsc{HiggsPredictions} frameworks and these are used for the following purpose:

\begin{itemize}
\item \textsc{HiggsPredictions} is used to determine theory production cross sections and decay rates.
\item \textsc{HiggsBounds} is used to find direct bounds for searches for new particles.
\item \textsc{HiggsSignals} is used to find the bounds from shifts to the observed Higgs boson's properties.
\end{itemize}

\textsc{HiggsBounds} and \textsc{HiggsSignals} contain a database of results from all key measurements from the LHC, LEP and other colliders.
The result from Chapter~\ref{sec:bsm_H_to_tau_tau_analysis} is included in this database.
The widths of additional Higgs bosons and branching fractions calculated with \textsc{2HDECAY} for Chapter~\ref{sec:H_A_to_4_tau_analysis} are utilised for the scan of the type X 2HDM parameter space.
The cross sections used in each analysis scanned over is scaled to that of the model parameters by the \textsc{NeutralEffectiveCouplings} function in \textsc{HiggsPredictions}.
A 95\% CL limit is then tested on each point in the parameter space. \\

In the alignment scenario, the strongest bounds on the type X \ac{2HDM} at all mass points tested come from the analysis described in Chapter~\ref{sec:bsm_H_to_tau_tau_analysis}.
The limits determined for the $m_{\phi}$ equal to 100 and 200 GeV scenarios are overlayed onto the limits shown in Figure~\ref{fig:4tau_md} and shown in Figure~\ref{fig:4tau_md_hb}. \\

\begin{figure}[!hbtp]
\centering
    \subfloat[]{\includegraphics[width=0.7\textwidth]{Figures/md_mphi100_hb.pdf}} \\
    \subfloat[]{\includegraphics[width=0.7\textwidth]{Figures/md_mphi200_hb.pdf}} 
\caption{Expected and observed 95\% CL exclusion contours on the $m_{A}$-$\tan\beta$ phase space in the type X 2HDM alignment scenario for $m_{\phi}$ scenarios of 100 GeV (a) and 200 GeV (b). The exclusion limit only on background expectation is shown as a dashed black line, the dark and bright grey bands show the 68\% and 95\% intervals of the expected exclusion and the observed exclusion contour is shown by the blue area. The limit obtained by \textsc{HiggsTools} is shown in the red contour.}
\label{fig:4tau_md_hb}
\end{figure}

The previous strongest constraints on the type X \ac{2HDM} alignment limit parameter space comes from the analysis described in Chapter~\ref{sec:bsm_H_to_tau_tau_analysis}.
The exclusion limits are at low values of $\tan\beta$ only.
As $\tan\beta$ increases, the gluon fusion and b associated production modes are suppressed but the branching ratios of the additional Higgs bosons to $\tau$ leptons are enhanced.
Therefore, the exclusion limit represents a compromise between suppressed cross sections and enhanced branching ratios. 
The regions where the product is large enough for that parameter point to be excluded, happens at low $\tan\beta$.
In the type X \ac{2HDM}, the b associated production mode in negligible due to no enhancement of couplings to b quarks.
The mass hypotheses change the limit due to the non-flat nature of Fig.~\ref{fig:model_independent_limits}(a).
The limit in Fig.~\ref{fig:4tau_md_hb}(b) is flat as the the strongest constraint comes from the $\phi$(H) boson, and for an additional neutral \ac{CP}-even Higgs boson $\tan\beta \lesssim 10$ is excluded.
However, this is not the case for Figure~\ref{fig:4tau_md_hb}(a) where $m_{\phi}$ is lighter and the sensitivity is not always driven by the $\phi$(h) boson.
In particular, the limit on a 100 GeV resonance, from Fig.~\ref{fig:model_independent_limits}, is weakest due to the large Drell-Yan background and the local excess observed on top of this peak, and so effects from both additional neutral Higgs bosons are present.
Together, the exclusion limits from both analyses yield an almost complete coverage of the type X 2HDM alignment scenario within the mass range searched. \\

Next the affect of the \ac{BSM} searches and precision measurements of the \ac{SM} Higgs boson on the type X \ac{2HDM} model, outside of the alignment scenario is studied.
Bounds are again calculated with \textsc{HiggsTools}, however the constraints now come from \ac{SM} Higgs measurements as well as \ac{BSM} searches such as described in Chapter~\ref{sec:bsm_H_to_tau_tau_analysis}.
The bounds determined, overlayed with the results shown in Figure~\ref{fig:4tau_cosbma}, are shown in Figure~\ref{fig:4tau_cosbma_hb}.
The constraints from the \ac{SM} Higgs boson, significantly narrow the region allowed outside of the alignment and motivate the continued use of alignment scenarios for extended Higgs sector searches.
The entirety of the phase space for the two mass points are shown are excluded by the combination of the searches detailed in Chapters~\ref{sec:bsm_H_to_tau_tau_analysis} and \ref{sec:H_A_to_4_tau_analysis}, as well as precision measurement of the \ac{SM} Higgs boson.

\begin{figure}[!hbtp]
\centering
    \subfloat[]{\includegraphics[width=0.65\textwidth]{Figures/csbma_phi200A100.pdf}} \\
    \subfloat[]{\includegraphics[width=0.65\textwidth]{Figures/csbma_phi200A160.pdf}} 
\caption{Expected and observed 95\% CL exclusion contours on the $\cos(\beta-\alpha)$-$\tan\beta$ phase space in the type X 2HDM alignment scenario with $m_{\phi}$ equal to 200 GeV and $m_{A}$ scenarios of 100 GeV (a) and 160 GeV (b). The exclusion limit only on background expectation is shown as a dashed black line, the dark and bright grey bands show the 68\% and 95\% intervals of the expected exclusion and the observed exclusion contour is shown by the blue area. The limit obtained by \textsc{HiggsTools} is shown in the red contour.}
\label{fig:4tau_cosbma_hb}
\end{figure}

\section{Outlook}

This thesis has presented two analyses utilising the full run 2 dataset collected by the \ac{CMS} experiment, from the 13 TeV proton proton collisions at the \ac{LHC}, targeting final states enriched in tau leptons.
The motivation for the searches presented in Chapter~\ref{sec:bsm_H_to_tau_tau_analysis} and \ref{sec:H_A_to_4_tau_analysis} are \ac{BSM} theories that attempt to resolve the theoretical issue of the hierarchy problem and the experiment tensions of the B anomalies and the g-2 anomaly.
The preceding chapters, act to motivate the new physics models that could potentially resolve these tensions, and the apparatus and methods used for the foundation of data taking and reconstruction required for the searches. \\

Chapter~\ref{sec:bsm_H_to_tau_tau_analysis} 