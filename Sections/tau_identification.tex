\section{Tau Identification}
\label{sec:tau_identification}

\begin{table}[h]
    \centering
    \begin{tabular}{cc}
         \hline
         Decay Mode & Branching Fraction  \\
         \hline
         \hline
         \textbf{Leptonic Decay (\(e\), \(\mu\))} & \textbf{35.2\%} \\
         \(e^- \bar{\nu}_e \nu_\tau \) & 17.8\% \\
         \(\mu^- \bar{\nu}_\mu \nu_\tau \) & 17.4\% \\
         \hline
         \textbf{Hadronic Decay (\(\tau_h\))} & \textbf{64.8\%} \\
         \(h^- \pi^0 \nu_\tau \) & 25.9\% \\
         \(h^- \nu_\tau\) & 11.5\% \\
         \(h^- 2\pi^0 \nu_\tau\) & 9.3\% \\
         \(\pi^- \pi^- \pi^+ \nu_\tau\) & 9.0\% \\
         \(\pi^- \pi^- \pi^+ \pi^0 \nu_\tau\) & 2.7\% \\
         other & 6.4\% \\
         \hline
    \end{tabular}
    \caption{Measured branching fractions, that are greater than 2\%, for the tau lepton. h represents a charged hadron either a pion or a kaon.}
    \label{tab:tau_decay}
\end{table}