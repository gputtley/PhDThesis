\section{\texorpdfstring{Vector Leptoquark Reinterpretation of the BSM H/A $\rightarrow \tau^+\tau^-$ Analysis}{Vector Leptoquark Reinterpretation of the BSM H/A to tau tau Analysis}}
\label{sec:vector_leptoquark_reinterpretation}

\subsection{Theoretical Motivation}

\begin{figure}[H]
\centering
\begin{tikzpicture}[scale=2]
  \begin{feynman}
    \vertex [label=left:$\bar{q}$] (a1) at (0,0);
    \vertex [label=left:$q$] (a2) at (0,1);
    \vertex (b1) at (1,0);
    \vertex (b2) at (1,1);
    \vertex [label=right:$U_{1}$] (b3) at (1,0.5);
    \vertex [label=right:$\tau^-$] (c1) at (2,0);
    \vertex [label=right:$\tau^+$] (c2) at (2,1);

    \diagram* {
      (b1) -- [fermion] (a1),
      (a2) -- [fermion] (b2),
      (b2) -- [boson] (b1),
      (c1) -- [fermion] (b1),
      (b2) -- [fermion] (c2),
    };
  \end{feynman}
\end{tikzpicture}
\caption{Feynman diagram showing the vector leptoquark t-channel interaction that produces a tau pair from a pair of bottom quarks.}
\label{fig:leptoquark_feynman}
\end{figure}

\subsection{Analysis Strategy}

\subsection{Signal Modelling}

\subsection{Results}

\subsection{Future Improvements to Vector Leptoquark Search}