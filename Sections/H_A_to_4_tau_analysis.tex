\chapter{\texorpdfstring{Search for New Physics in $\tau^+\tau^-\tau^+\tau^-$ Final States}{Search for new physics in tautautautau final states}}
\label{sec:H_A_to_4_tau_analysis}

Enhancement from $\tan\beta$ to up or down-like quark couplings to additional Higgs bosons are essential for the majority of searches for extended Higgs sectors, including in the analysis detailed in Chapter~\ref{sec:bsm_H_to_tau_tau_analysis}.
However, in some 2HDMs it can be the case that both up and down-like couplings to additional Higgs bosons are suppressed and this parameter space is left relatively untouched by "MSSM-like" searches.
In particular, type X 2HDMs where only lepton couplings are enhanced by $\tan\beta$, allow for new physics loop contributions to SM measurements through couplings between leptons and additional Higgs bosons.
This is particularly interesting in the context of the g-2 anomaly \cite{} with reasoning explained in Section~\ref{}.
This chapter will detail a search for such an extended Higgs sector, that looks for a production mode not suppressed at high $\tan\beta$, through the process $Z^{*}\rightarrow \phi A \rightarrow 4\tau$.
This search is split up into two sections:

\begin{enumerate}[i)]
  \item A model independent search for the $Z^{*}\rightarrow \phi A \rightarrow 4\tau$ process. Both additional particles are required to have narrow width and no assumptions are made on the production cross-section via an off-shell Z boson or the branching fraction of $phi$ and A decaying to a pair of tau leptons.
   \item A search for the type X 2HDM, motivated by the phase space for possible explanations to the g-2 anomaly. The $m_{A}$-$\tan\beta$ phase space for scenarios of $m_\phi$ in the alignment limit is scanned, as well as checks outside of this limit on the $\cos(\beta-\alpha)$-$\tan\beta$ for specific scenarios of both $m_\phi$ and $m_A$.
\end{enumerate}

These searches are performed with the full run-2 dataset ($138 \sifb$) collected by the CMS experiment. 

\section{Signal Modelling}

Any additional Higgs boson produced in the type X 2HDM at high $\tan\beta$ will predominantly decay to tau leptons.
To probe the type X 2HDM at high $\tan\beta$, a production process that is not suppressed is required.
Ref.~\cite{} discusses that the following production modes of two additional Higgs bosons are dominant to produce any of these new particles at high $\tan\beta$:
\begin{enumerate}[i)]
  \item $pp \rightarrow Z^{*} \rightarrow \phi A \rightarrow (\tau^{-}\tau^{+})(\tau^{-}\tau^{+})$
  \item $pp \rightarrow Z^{*} \rightarrow H^{+}H^{-} \rightarrow (\tau^{-}\nu)(\tau^{+}\nu)$
  \item $pp \rightarrow W^{\pm *} \rightarrow H^{\pm}A \rightarrow (\tau^{\pm}\nu)(\tau^{-}\tau^{+})$
  \item $pp \rightarrow W^{\pm *} \rightarrow H^{\pm}\phi \rightarrow (\tau^{\pm}\nu)(\tau^{-}\tau^{+})$
\end{enumerate}
As the production cross sections are of the similar magnitudes, the search sensitivities depend on the separation of the signals from background.
In general, the more objects you can select in the final state, the smaller the background contributions.
This is certainly true in tau enriched final states, where backgrounds can be dominated jets misidentified as hadronic taus and so every extra tau selected reduces this background.
In particular, (ii) has the production cross sections \cite{} far smaller than the observed limit for gluon fusion production of a single resonance shown in Figure~\ref{fig:model_independent_limits}(a) and it is not possible to use tau decay product and MET alignment to separate the background, so does seem not a viable search option with the run-2 CMS dataset.
The increased background from fewer object selections and looser charge sum selection on (iii) and (iv), makes (i) the golden search channel for a type X 2HDM. 
A Feynman diagram for this process is shown in Figure~\ref{fig:4tau_feynamn}. \\

\begin{figure}[H]
\centering
\begin{tikzpicture}[scale=2]
  \begin{feynman}
    \vertex [label=left:$q$] (a1) at (0,-0.25);
    \vertex [label=left:$\bar{q}$] (a2) at (0,1.25);
    \vertex (b) at (0.7,0.5);
    \vertex [label=above:$Z^{*}$] (b1) at (1.05,0.5);    
    \vertex (c) at (1.4,0.5);
    \vertex [label=below:$h/H$] (d11) at (1.75,0.15);
    \vertex [label=above:$A$] (d12) at (1.75,0.85);
    \vertex (d1) at (2.1,0);
    \vertex (d2) at (2.1,1);
    \vertex [label=right:$\tau^-$] (e1) at (2.7,-0.25);
    \vertex [label=right:$\tau^+$] (e2) at (2.7,0.25);
    \vertex [label=right:$\tau^-$] (e3) at (2.7,0.75);
    \vertex [label=right:$\tau^+$] (e4) at (2.7,1.25);
    \diagram* {
      (a1) -- [fermion] (b),
      (b) -- [fermion] (a2),
      (b) -- [photon] (c),
      (c) -- [scalar] (d1),
      (c) -- [scalar] (d2),
      (d1) -- [fermion] (e1),
      (e2) -- [fermion] (d1),
      (d2) -- [fermion] (e3),
      (e4) -- [fermion] (d2),
    };
  \end{feynman}
\end{tikzpicture}
\vspace*{10mm}
\caption{Diagram of production of two additional neutral Higgs bosons from an off-shell Z boson and their decays to tau leptons.}
\label{fig:4tau_feynamn}
\end{figure}

SIGNAL GENERATION \\

BRANCHING FRACTIONS \\


\section{Event Selection}

\begin{table}[h]
    \centering
    \begin{tabular}{|c|c|}
         \hline
         Channel & Branching Fraction  \\
         \hline
         \hline
         $e \tau_h \tau_h \tau_h$ & 19.4\% \\
         $\mu \tau_h \tau_h \tau_h$ & 18.9\% \\
         $\tau_h \tau_h \tau_h \tau_h$ & 17.6\% \\
         $e \mu \tau_h \tau_h$ & 15.6\% \\
         $e e \tau_h \tau_h$ & 8.0\% \\
         $\mu \mu \tau_h \tau_h$ & 7.6\% \\
         $e e \mu \tau_h$ & 4.3\% \\
         $e \mu \mu \tau_h$ & 4.2\% \\
         $e e e \tau_h$ & 1.5\% \\
         $\mu \mu \mu \tau_h$ & 1.4\% \\
         $e e e \mu$ & 1.4\% \\
         $e e \mu \mu$ & 0.6\% \\
         $e \mu \mu \mu$ & 0.4\% \\
         $e e e e$ & 0.1\% \\
         $\mu \mu \mu \mu$ & 0.1\% \\
         \hline
    \end{tabular}
    \caption{}
\end{table}


\section{Signal Extraction}
\section{Background Modelling}
\subsection{Overview}
\subsection{Machine Learning Fake Factor Method}
\section{Corrections}
\section{Results}
