\begin{titlepage}

\newcommand{\HRule}{\rule{\linewidth}{0.5mm}} % Defines a new command for the horizontal lines, change thickness here

%----------------------------------------------------------------------------------------
%	LOGO SECTION
%----------------------------------------------------------------------------------------

\includegraphics[width=8cm]{title/logo.eps}\\[1cm] % Include a department/university logo - this will require the graphicx package
 
%----------------------------------------------------------------------------------------

%\center % Center everything on the page
\begin{center}

%----------------------------------------------------------------------------------------
%	HEADING SECTIONS
%----------------------------------------------------------------------------------------

%\textsc{\LARGE PhD 9M Early Stage Assessment Report}\\[1.5cm] % Name of your university/college
%\textsc{\Large Imperial College London}\\[0.5cm] % Major heading such as course name
%\textsc{\large Department of Physics}\\[0.5cm] % Minor heading such as course title

%----------------------------------------------------------------------------------------
%	TITLE SECTION
%----------------------------------------------------------------------------------------
\makeatletter
\HRule \\[0.4cm]
{ \huge \bfseries \@title}\\[0.4cm] % Title of your document
\HRule \\[1.5cm]
 
%----------------------------------------------------------------------------------------
%	AUTHOR SECTION
%----------------------------------------------------------------------------------------

%\begin{minipage}{0.4\textwidth}
%\begin{flushleft} \large
%\emph{Author:}\\
%\@author % Your name
%\end{flushleft}
%\end{minipage}
%~
%\begin{minipage}{0.4\textwidth}
%\begin{flushright} \large
%\emph{Supervisor:} \\
%Prof. David Colling \\[1.2em] % Supervisor's Name
%\end{flushright}
%\end{minipage}\\[2cm]
%\makeatother

\begin{centering}
\Large \@author \\
\vspace{0.5cm}
\Large Imperial College London \\
\Large Department of Physics \\
\vspace{8cm}
\small A thesis submitted to Imperial College London \\
\small for the degree of Doctor of Philosophy
\end{centering}

\end{center}

\newpage

\thispagestyle{empty}
\mbox{}

\newpage

\vspace*{\fill}
The copyright of this thesis rests with the author and is made available under a Creative Commons Attribution Non-Commercial No Derivatives licence. 
Researchers are free to copy, distribute or transmit the thesis on the condition that they attribute it, that they do not use it for commercial purposes and that they do not alter, transform or build upon it. 
For any reuse or redistribution, researchers must make clear to others the licence terms of this work.
\vspace*{\fill}

\newpage

\vspace*{\fill}
\begin{adjustwidth}{1cm}{1cm}
\begin{center}
\Large \textbf{Abstract}
\vspace{0.5cm}
\end{center}

The Standard Model (SM) of particle physics is currently the best model of the fundamental particles and their interactions. 
However, there are significant theoretical issues and experimental tensions with the model. 
The theoretical issues include the hierarchy problem which forecasts the breakdown of the SM when looking at the size of corrections needed to calculate the mass of the newest found member of the theory, the Higgs boson particle. 
The current experimental tensions include the B anomalies and the measurement of the muon anomalous magnetic moment. 
Looking for signatures of theoretical explanations to these issues offers excellent search options for new fundamental particles. 
This thesis describes searches for new physics that can explain both the theoretical problems and experimental tensions. 
This is done using tau ($\tau$) leptons seen during Run 2 of the Large Hadron Collider at the Compact Muon Solenoid (CMS) experiment. 
The beyond SM theories searched for range from the Minimal Supersymmetric SM (MSSM), leptoquarks, and the type X two Higgs doublet models (2HDM). 
Two local excesses are observed with statistical significances $\approx 3\sigma$ for a resonance produced via gluon fusion in the $\tau\tau$ final states, at masses of 100 GeV and 1.2 TeV. 
Otherwise, good agreement is observed between the SM background prediction and data.
Leading constraints are placed on scenarios of the MSSM phase space and limits that shrink the allowed explanation to the B anomalies are set.
The type X 2HDM, as an explanation for the muon anomalous magnetic moment measurement, is fully ruled out and many mass hypotheses in the type X 2HDM are completely excluded by this thesis.
\end{adjustwidth}
\vspace*{\fill}
\newpage

\vspace*{\fill}
\begin{adjustwidth}{1cm}{1cm}
\begin{center}
\Large \textbf{Declaration}
\vspace{0.5cm}
\end{center}

I declare that the work in this thesis is mine. 
Figures and results taken from other sources are indicated by a reference in the text or figure caption. 
Figures labelled \say{CMS} are sourced from CMS publications. 
Figures labelled \say{CMS Supplementary} have been made public as supplementary material accompanying a publication. 
The \say{CMS Preliminary} label is given to CMS public results that are yet to be finalised by the collaboration and the journal.
The \say{Simulation} tag is added when only simulated data is used to generate a CMS public plot.
All figures with these labels, including those made by myself, also include the relevant reference in the caption.
The analyses presented in this document were developed in collaboration with other members of the CMS experiment. 
Chapters~\ref{sec:theory}--\ref{sec:object_reconstruction} do not contain original work by myself but are intended to describe the theoretical motivation, as well as the apparatus and methods used to generate and collect the data, that are critical to the work presented in the following chapters. 
The high-mass additional Higgs boson search, described in Chapter~\ref{sec:bsm_H_to_tau_tau_analysis}, and the origin of the background modelling methods were pre-established in Reference~\cite{CMS_MSSM_Tau_2018}.
The work described in Chapter~\ref{sec:bsm_H_to_tau_tau_analysis} was done in collaboration with the other members of the CMS $H\rightarrow\tau\tau$ group.
For the additional Higgs boson component of this chapter, I contributed to the low-mass optimisation, derivation and parametrisation of fake factors, evaluation of uncertainties and the final statistical fits to data.
The vector leptoquark interpretation of the results was solely my work.
The results in Chapter~\ref{sec:bsm_H_to_tau_tau_analysis} were made public in Reference~\cite{CMS:2022rbd}.
Chapter~\ref{sec:H_A_to_4_tau_analysis} contains work performed in collaboration with the Imperial College London $H\rightarrow\tau\tau$ group.
I was responsible for the full workflow of this analysis including signal simulation, background modelling, optimisation, uncertainty modelling, interpretations and the final statistical fits to data.
This analysis is currently not published, but a publication is planned in the near future.
Chapter~\ref{sec:conclusion} includes interpretations of the results in Chapters~\ref{sec:bsm_H_to_tau_tau_analysis} and \ref{sec:H_A_to_4_tau_analysis}, that are entirely my work.

\begin{FlushRight}
George Uttley
\end{FlushRight}
\end{adjustwidth}
\vspace*{\fill}

\newpage


\vspace*{\fill}
\begin{adjustwidth}{1cm}{1cm}
\begin{center}
\Large \textbf{Acknowledgements}
\vspace{0.5cm}
\end{center}

I would like to thank the Imperial College London High Energy Physics group and the Science and Technology Facilities Council for giving me the opportunity to conduct this research.
Thank you to my supervisor, David Colling, for all the assistance and guidance, in particular, the aid I received whilst working through the pandemic.
I also owe a massive thank you to Daniel Winterbottom, for all of the help and advice, as well as the patience he showed with me throughout my PhD.
I am also grateful to the remaining members of the $H\rightarrow\tau\tau$ group for all of the interesting discussions and continued motivation whilst writing this thesis.
Thank you to my friends and family, who have continually supported me throughout my life and are the reason I was able to complete this work.
In particular, thank you to Emmy for her unwavering support and encouragement throughout this journey.

\begin{FlushRight}
George Uttley
\end{FlushRight}
\end{adjustwidth}
\vspace*{\fill}

\end{titlepage}