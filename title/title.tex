\begin{titlepage}

\newcommand{\HRule}{\rule{\linewidth}{0.5mm}} % Defines a new command for the horizontal lines, change thickness here

%----------------------------------------------------------------------------------------
%	LOGO SECTION
%----------------------------------------------------------------------------------------

\includegraphics[width=8cm]{title/logo.eps}\\[1cm] % Include a department/university logo - this will require the graphicx package
 
%----------------------------------------------------------------------------------------

%\center % Center everything on the page
\begin{center}

%----------------------------------------------------------------------------------------
%	HEADING SECTIONS
%----------------------------------------------------------------------------------------

%\textsc{\LARGE PhD 9M Early Stage Assessment Report}\\[1.5cm] % Name of your university/college
%\textsc{\Large Imperial College London}\\[0.5cm] % Major heading such as course name
%\textsc{\large Department of Physics}\\[0.5cm] % Minor heading such as course title

%----------------------------------------------------------------------------------------
%	TITLE SECTION
%----------------------------------------------------------------------------------------
\makeatletter
\HRule \\[0.4cm]
{ \huge \bfseries \@title}\\[0.4cm] % Title of your document
\HRule \\[1.5cm]
 
%----------------------------------------------------------------------------------------
%	AUTHOR SECTION
%----------------------------------------------------------------------------------------

%\begin{minipage}{0.4\textwidth}
%\begin{flushleft} \large
%\emph{Author:}\\
%\@author % Your name
%\end{flushleft}
%\end{minipage}
%~
%\begin{minipage}{0.4\textwidth}
%\begin{flushright} \large
%\emph{Supervisor:} \\
%Prof. David Colling \\[1.2em] % Supervisor's Name
%\end{flushright}
%\end{minipage}\\[2cm]
%\makeatother

\begin{centering}
\Large \@author \\
\vspace{0.5cm}
\Large Imperial College London \\
\Large Department of Physics \\
\vspace{8cm}
\small A thesis submitted to Imperial College London \\
\small for the degree of Doctor of Philosophy
\end{centering}

\end{center}

\newpage

\begin{center}
\Large \textbf{Abstract}
\vspace{0.5cm}
\end{center}

The Standard Model of Particle Physics is currently the best model of the fundamental particles and their interactions. However, there are still significant theoretical issues and recently seen experimental tensions with the model. The theoretical issues include the hierarchy problem which forecasts the breakdown of the Standard Model when looking at the size of corrections needed to calculate the mass of the newest found member of the theory, the Higgs boson particle. The current experimental tensions include the B-anomalies and the g-2 measurement. These results, although they do not yet sit at the required 5$\sigma$ deviation for a discovery, offer the most prominent leads into where new physics may be hiding. Looking for signatures of theoretical explanations of these anomalies offers excellent search options for new fundamental particles. This thesis describes the search for new physics that can explain both the theoretical problems and experimental tensions. This is done using tau leptons seen during Run-2 of the Large Hadron Collider (LHC) at the Compact Muon Solenoid (CMS) experiment. The Beyond Standard Model theories searched for range from Supersymmetry, leptoquarks, to type-X two Higgs doublet models. Each theory is separately studied and an analysis is tailored to find its most sensitive signature. In the process of optimisation, data-driven background modelling is improved to aid the reliability of the results. The results are currently blinded however the expected limits offer some of the largest constraints that are placed on these prominent Beyond Standard Model Theories.

\newpage

\begin{center}
\Large \textbf{Declaration}
\vspace{0.5cm}
\end{center}

I did this work I promise

\newpage

\begin{center}
\Large \textbf{Acknowledgements}
\vspace{0.5cm}
\end{center}

Emmeline

\end{titlepage}